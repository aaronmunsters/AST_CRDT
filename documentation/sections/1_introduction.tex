\section{Introduction}\label{sec:introduction}

With telecommuting on the rise, collaboratively working on a single code base in real-time is gaining traction.
The code editor Visual Studio Code, which is the current leading choice of IDE\footnote{As is reported by the
% https://insights.stackoverflow.com/survey/2021#section-most-popular-technologies-integrated-development-environment
\href{
    https://insights.stackoverflow.com/survey/2021\#section-most-popular-technologies-integrated-development-environment
}{2021 Stack Overflow Developer Survey}}, promotes the
\href{https://marketplace.visualstudio.com/items?itemName=MS-vsliveshare.vsliveshare}{Live Share extension}
as part of its top 50 most installed extensions\footnote{As was displayed on the
\href{https://marketplace.visualstudio.com/search?target=VSCode\&category=All\%20categories\&sortBy=Installs}
{Visual Studio Marketplace most popular extensions} as of 7 January 2022}, demonstrating the relevance of providing
developers a structured basis to develop in a collaborated manner.

Traditional collaborating tools such as Live Share for Visual Studio Code or Teletype for Atom in essence enable
collaboration by representing the code files that are being worked on as large strings which are concurrently
manipulated by different replica's, ie\. the machines on which remote developers work.
This technique can be considered a simplified approach to enabling the replication of the code, as code usually takes
a strict structural form, which is then completely overlooked by this technique.
This brings with it different limitations.

\begin{itemize}
    \item \textbf{Easy to break the code structure} \\
    When developers write code alone, they tend to make syntactic mistakes.
    Even though these mistakes are pointed at by the parser, it is not always an easy task to locate the culprit of
    this case, for example identifying the location at which a matching bracket is missing to correct the program
    both syntactically and semantically can get complex for code with lots of nested statements.
    When it is up to two developers to locate the missing matching bracket, parallel development only complicates
    matters to identify who is to blame and what part of the code to repair for such breaking changes.

    \item \textbf{Need for strong synchronization} \\
    Traditional collaboration tools enable collaboration by assigning one developer as the host and inviting
    others to join their development sessions\footnote{
        Live Share calls these \emph{session}s, Teletype calls these \emph{portal}s.}
    which requires the host to stay live at all times.
    The host as such could be considered a single point of failure, which may break the application availability upon
    the occurrence of a network partition.

    \item \textbf{Weak merge strategy:}
    In case the collaboration tool opts for availability by enabling the developer to continue on a local copy of
    the code during a network partition, upon a connection reestablishment the tool is uninformed about the code and its
    structure on how it should be merged.

    It could be considered an easy task to merge little edits, say the addition of a single character.
    Concurrent changes to the overall program structure, however, such as statement moves and updates can significantly
    add to the complexity of such merges.
    Figure~\ref{fig:scheme-code-evolving} illustrates this with a concrete example.
    The initial code contains two statements, an \mintinline{scheme}{eat} and a \mintinline{scheme}{touch} statement.
    For a given network partition, the local edit of changing the \mintinline{scheme}{"apple"} string into the string
    \mintinline{scheme}{"banana"} should take place once the connection is reestablished.
    In case a different remote replica changes the order of the statements, should result in both the update and the
    move operation having taken place.
    This complexity of merging both operations increases with the size of the code that is being worked on such as
    tracking the moved statement across more than neighbouring lines or even in different scopes.
\end{itemize}

\begin{figure}
    \centering
    \includegraphics[width=\columnwidth]{images/out/evolving_code}
    \caption{
        A sample Scheme program that evolves through different machines that temporarily break synchronization.
        This example shows that reasoning over the code structure is required to successfully merge the two separate
        operations.
    }
    \label{fig:scheme-code-evolving}
\end{figure}

\begin{sloppypar}
    To overcome these limitations this work proposes a novel strategy to provide developers with an editor that models
    the takes underlying AST that is being edited concurrently through modelling the tree as a conflict-free replicated
    tree.
    The source code for the project is available on
    \href{https://github.com/aaronmunsters/AST\_CRDT}{https://github.com/aaronmunsters/AST\_CRDT} while the working
    prototype is live available on
    \href{https://aaronmunsters.github.io/AST\_CRDT/}{https://aaronmunsters.github.io/AST\_CRDT/}.
    This report is structured to motivate the new approach in
    Section~\ref{sec:replicated-conflict-free-abstract-syntax-tree} while covering technical challenges in
    Sections~\ref{sec:implementing-the-abstract-syntax-tree} and~\ref{sec:computing-ast-changes}.
    As a small demonstration application is included with this report, as it is discussed in
    Section~\ref{sec:integrated-demo}.
    Section~\ref{sec:tests} discusses the set of tests included in the codebase to assert the correct behaviour of the
    code, while Section~\ref{sec:future-work} covers limitations known to the work and proposes improvements in the form
    of future work.
\end{sloppypar}
