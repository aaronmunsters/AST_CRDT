\section{Computing AST Changes}\label{sec:computing-ast-changes}

Before propagating AST changes over the network, they need to be determined locally.
Computing source-code changes can be performed by incremental compilers, they could be derived from projectional
editors\footnote{These are editors that let the user develop their programs through manipulating an AST directly
rather than having the programmer type the program in an open space for characters.
The benefit of these editors is that the programmer can make no syntactic mistakes, the drawback is that the amount of
different building blocks to construct a program can become complex for languages with a rich syntax.
An example projectional editor that is actively maintained for the TypeScript language is
\href{https://github.com/tehwalris/forest}{Forest}.
}, or they can be computed by AST tree differencing algorithms\footnote{For the reader, a curated list of papers and
tools is available on the page \href{https://www.monperrus.net/martin/tree-differencing}{"Pointers on abstract syntax tree
differencing algorithms and tools"}.}.

Whereas incremental compilers and projectional editors come with the disadvantage that they are specialized per
language, AST tree differencing algorithms exist with \emph{language agnostic} specifications.
This project has adopted the GumTree algorithm~\cite{10.1145/2642937.2642982}, a language-agnostic algorithm to compute
the changes between two ASTs.

Computing changes in trees produces an \emph{edit script}, which is an ordered set of edit operations that when applied to
the source tree results in an output of the destination tree.
The operations that suffice for tree transformations are captured by the following operations~\cite{10.1145/235968.233366}:
\begin{itemize}
    \item \textit{update(n, v)} to update the content of node \textit{n} with the value of \textit{v}.

    \item \textit{add(t, p, i, l, v)} to add the node \textit{t} with the label \textit{l} and the value \textit{v} to
    the parent \textit{p} at index \textit{i}.

    \item \textit{delete(t)} to delete node \textit{t}.

    \item \textit{move(t, p, i)} to move node \textit{t} in the tree to be a child of \textit{p} at index \textit{i}.
\end{itemize}
When spoken of \emph{minimum edit script}, the edit script is the smallest possible ordering of edit operations that is
still a valid edit script.
In the context of the CRDT AST, the need for remaining as close as possible to a minimum edit script is desirable since
it can affect the network load.

The GumTree algorithm consists of a top-down phase and bottom-up phase to compute a set of mappings between both ASTs,
while it leaves the algorithm to determine an edit script open to the implementor.
The GumTree implementation for this work\footnote{
    The authors of GumTree provided their implementation as a publicly available
    \href{https://github.com/GumTreeDiff/gumtree}{GitHub project}, however, it is implemented using the Java language and is
    not developed with compilation for a web browser in mind.}
to compute the edit script is based on the work of~\citet{10.1145/235968.233366}.

The class \mintinline{scala}{MinimumEditScript} implements the \emph{EditScript} algorithm~\cite{10.1145/235968.233366},
while the class \mintinline{scala}{GumTreeAlgorithm} implements the \emph{Gumtree}
algorithm~\cite{10.1145/2642937.2642982}.
These implementations remain true to the specification of the author.
