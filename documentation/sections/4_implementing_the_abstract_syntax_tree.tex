\section{Implementing the Abstract Syntax Tree}\label{sec:implementing-the-abstract-syntax-tree}

The implementation of the conflict-free replicated abstract syntax tree can be viewed from two perspectives.
On one hand, the implementation responsible for respecting the requirements of a CRDT, discussed in
Section~\ref{subsec:implementing-the-crdt}, and on the other hand the implementation for the underlying locally kept
data structure, discussed in Section~\ref{subsec:implementing-the-local-ast}.

\subsection{Implementing the CRDT}\label{subsec:implementing-the-crdt}

The mentioned motivation in Section~\ref{sec:replicated-conflict-free-abstract-syntax-tree} forms the basis of the
\mintinline{scala}{ConflicFreeReplicatedAst} class.
Objects for this class are created with an object that implements the \mintinline{scala}{Transmitter} trait.
This trait guarantees the definition of a \mintinline{scala}{publish} method to publish new edit scripts and a
\mintinline{scala}{subscribe} method to install a callback to receive incoming edit scripts.

The \mintinline{scala}{ConflicFreeReplicatedAst} class implements the CRDT methods \mintinline{scala}{update} for
local replica changes, \mintinline{scala}{query} to query for the local data structure and \mintinline{scala}{merge}
for incoming operations.
Objects for this class locally keep a variable \mintinline{scala}{operations} which contains the aggregation of all
locally produced and received \emph{edit script}s that can be ordered by their timestamp.

Upon receiving local updates through \mintinline{scala}{update}, it invokes the transmitter's
\mintinline{scala}{publish} method to propagate the changes to other replica's.
The \mintinline{scala}{query} method sorts the operations and folds them starting an empty AST,
happening in a non-deterministic manner to guarantee the same set of operations across replica's results in the same
AST.
The \mintinline{scala}{merge} method adds the incoming operations to the locally stored operations.

\subsection{Implementing the local AST}\label{subsec:implementing-the-local-ast}

The abstract syntax tree itself is implemented by the \mintinline{scala}{HeadedAST} case class, which consists of an
optional root (in case the AST is empty) and a mapping from identities to nodes, called its \emph{header}.
This class does not assume a type of identity used to identify nodes, as it is implemented through generics.
This case class, as most of the codebase, is implemented in a functional style.
The choice for doing so, as it was the first class on which work began, was to ease the feature to perform the rollback
as through functional code the application of edits in the same order across all replica's would always yield the
same outcome AST.
Additionally, the code is easier to test.

The internal structure of the AST, consisting of a mapping and an optional root, would allow for easy serialization of
the AST by serializing the root and the nodes.
The nodes of the AST each implementing \mintinline{scala}{SchemeNode} trait, covering generic syntax node operations
such as querying for the parent, the depth of the tree and such.
A more intuitive implementation for the AST would be to implement the tree structure as a root with pointers to other
nodes (that in their turn point to other nodes) rather than keeping a mapping in the AST from identity to the nodes.
Having opted use of this indirection through the AST \emph{header} makes it trivial to serialize AST nodes, as they
thus never consist of pointers to values in memory (which is critical in a distributed setting since replica's
classically do not share a memory-space).
Serialization is left to the \href{https://github.com/suzaku-io/boopickle}{BooPickle library} such that it does
not add to the complexity of the code, which is developed with high performance in mind~\cite{10.1145/2509136.2509547}.

The nodes of the Scheme-like language that has been implemented are limited to \emph{number}s, \emph{string}s,
\emph{identifier}s and \emph{expression}s.
This set of nodes is relatively small but is sufficiently large to provide a prototype that covers all source-code edit
operations for a language supporting primitive values and nodes with descendants, showing support for arbitrary nested
setups.

Turning a textual representation of a program into its \mintinline{scala}{HeadedAST} representation is made possible
by the \mintinline{scala}{Parser} object, which requires an identity generator function to assign each node with a new
identity, which parses the programs using the grammar developed in combination with the
\href{https://com-lihaoyi.github.io/fastparse/}{FastParse library}.
