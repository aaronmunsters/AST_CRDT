\section{Future Work}\label{sec:future-work}

Different improvements for the current implementation are discussed below.

\begin{itemize}
    \item \textbf{Arbitrary order of operations.}\\
    A limitation for the current implementation is that the merge strategy performs an ordering of the concurrent
    operations in an arbitrary order (in the prototype this is based on the replica identity), though deterministic
    there is no meaning as to which replica performs the winning operations\footnote{
        The mentioned 'no winning operation' statement holds only for concurrent operations on nodes that are already
        present on both replica's, say one replica updates a literal number to negative infinity while concurrently
        another replica updates the value to positive infinity, the result actual result will be an arbitrary operation
        based on the replica identities.
        For the addition of new nodes, no issues arise.
    }.

    An improvement hereto could be to assign developer roles through the user interface, allowing to sort the operations
    in an order that the hierarchically higher roles their operations are prioritized to win over other operations.
    An alternative improvement could be for the user interface to prompt the user to decide which action should
    dominate.

    \item \textbf{Merge granularity.}\\
    The merges take place on a granularity of happening on a per-node basis.
    Though sufficiently powerful, different merging strategies could be desired based on the types of nodes that are
    being affected, two examples are provided.

    For string literal nodes, two concurrent updates will only take into account the final operation after the sort to
    have taken place, while it would be more optimal to delegate the update operations to the string node to further
    apply the operations that are appropriate to merge text, similar to text editors.

    For literal set nodes, two concurrent additions of an equal value would duplicate the content, while this does not
    make sense for set nodes thus it would be the desired effect to prevent the double addition of the content.
\end{itemize}
