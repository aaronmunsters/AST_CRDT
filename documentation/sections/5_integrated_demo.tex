\section{Integrated Demo}\label{sec:integrated-demo}

The project comes included with a working prototype.
The prototype uses the \mintinline{scala}{ConflictFreeReplicatedIntAst} class which extends the
\mintinline{scala}{ConflictFreeReplicatedAst} class.
It provides concrete types for the \mintinline{scala}{Identity} generic (as a pair of the replica identity combined
with a unique integer) and the \mintinline{scala}{EditIdentity} generic (as a pair of a Lamport clock in combination
with a replica identity) as Listing~\ref{code:instantiation-for-demo} shows\footnote{
    By having the case classes extend \mintinline{scala}{AnyVal}, the compiler handles these as value classes.
    This allows for compile-time safety to not mix up types while keeping a low memory footprint by representing the
    values as integers at runtime.
}.

\begin{listing}
    \begin{minted}[fontsize=\footnotesize]{scala}
// file: ReplicatedIntOp.scala
case class NId(id: Int) extends AnyVal
case class LClock(count: Int) extends AnyVal
case class RId(identity: Int) extends AnyVal

// file: ConflictFreeReplicatedAst.scala
class ConflictFreeReplicatedAst[
    Identity,
    EditIdentity
](...){...}

// file: ConflictFreeReplicatedAst.scala
case class ConflictFreeReplicatedIntAst(...)
    extends ConflictFreeReplicatedAst[
        (RId, NId),
        (LClock, RId)
    ](...){...}
    \end{minted}
    \caption{
        Code snippets across three files demonstrating how the \mintinline{scala}{ConflictFreeReplicatedAst} is
        specialized for the demo.
        The identity of AST nodes is represented as a pair of a replica identity and a node identity.
        The identity of edit operations is represented as a pair of a Lamport clock and a replica identity.
    }
    \label{code:instantiation-for-demo}
\end{listing}

\begin{sloppypar}
    The \mintinline{scala}{ConflictFreeReplicatedIntAst} class and the \mintinline{scala}{ReplicatedIntOp} class are the
    dependencies used by the \mintinline{scala}{CRDT_AST_ScalaJS} object, which is the interfacing file with the JavaScript
    codebase that makes the demo possible in a web browser.
\end{sloppypar}

The demo uses the \href{https://microsoft.github.io/monaco-editor/}{Monaco Editor}, which is the same editor
that is used in the VS Code editor.
Through a correct configuration during setup this enables syntax highlighting for Scheme-like languages.
Throughout the lifetime of the code whenever the code is changed the AST is kept formatted in the browser to ensure
all replica's with the same operations have the same view.

To enable the propagation of operations for the operation-based CRDT, the demo uses the
\href{https://peerjs.com/}{PeerJS library}.
This is a JavaScript library that allows setting up peer-to-peer connections between different clients on the web using
the WebRTC standard.
This network setup is used but can be interchanged with other configurations such as a client-server setup.

Through the use of the \href{https://github.com/suzaku-io/boopickle}{BooPickle library}, the operations that need to be
sent over the wire are serialized to bytes before they are transmitted to other peers.
By doing so rather than sending textual representations the demo aims to maintain a low network load.
