\section{A Replicated Conflict Free Abstract Syntax Tree}\label{sec:replicated-conflict-free-abstract-syntax-tree}

This work proposes a new approach to collaborative code editing through modelling the underlying abstract syntax
tree (AST) as a conflict-free replicated datatype~\cite{10.1007/978-3-642-24550-3_29} (CRDT). % TODO: find the first use of AST and CRDT and add the abbrevs
The prototype accompanied by this report is an implementation for the Scheme language.

Figure~\ref{fig:platform-architecture} provides an overview of the flow that a single local edit makes through the
different stages, each associated with a concrete component, before it enables all replica's to converge to an identical
AST.
The approach consists of the code editor continuously monitoring (in an event-based manner) for changes taking place on
the local replica.
For incoming changes in the editor, the altered source code is parsed and the newly constructed AST is mapped onto the
previous AST, after which the delta for both ASTs can be computed in the form of an edit script, meaning a script that
contains the changes to go from the source AST to the destination AST.
This edit script containing the changes is then propagated to all other replica's accompanied by a logical timestamp,
forming the basis of the AST as an operation-based CRDT.
For the replica's, each incoming edit script is added to its local list of operations, which can be reordered due to the
inclusion of the logical timestamp, allowing for an ordering of the events.
The ordered list of edits allows the local replica to compute the final resulting AST, which is then reflected in
the code editor.

The project is developed in the Scala language, built using structures and libraries that are ScalaJS compliant.
This comes with the benefits of the scala type checker asserting type safety while remaining portable in terms of
the underlying execution engine, which can be either the JVM or a JavaScript engine.

%%%%%%%%%%%%%%%%%%%%%%%%%%%%%%%%%%%%%%
%%%%%%% Alternative strategies %%%%%%%
%%%%%%%%%%%%%%%%%%%%%%%%%%%%%%%%%%%%%%
\begin{tcolorbox}[title=Approach motivation]
    \label{example:racecondition}
    Initial work was put into remaining language agnostic, where the AST changes could be detected by an
    incremental parser such as \href{https://www.npmjs.com/package/escaya}{Escaya} for JavaScript or
    \href{https://ts-morph.com}{ts-morph} for TypeScript such that the reported changes would be propagated to other
    replica's.

    This strategy was rejected as these incremental parsers are built with a single local in-memory AST in mind, where
    the increments are reported in terms of pointers to AST nodes that have been affected, which would require in-depth
    adjustments to providing each node with unique identifiers and changing the incremental parsers to report
    serializable changes that can be shared across the network to devices with isolated memory.

    A different strategy thus was opted for to implement a small parser for a Scheme-like language,
    allowing to remain in full control of the internal representations, opting for a serializable-first design for the
    AST and the detected changes.
\end{tcolorbox}

%%%%%%%%%%%%%%%%%%%%%%%%%%%%%%%%%%%%%%%%%%%%%%%%%%%%%%%%%%%%%%%%%%%%%%%%%%%%%%%%
%%%%%%% An overview of the stages that flow through different components %%%%%%%
%%%%%%%%%%%%%%%%%%%%%%%%%%%%%%%%%%%%%%%%%%%%%%%%%%%%%%%%%%%%%%%%%%%%%%%%%%%%%%%%

\begin{figure*}
    \centering
    \includegraphics[width=\textwidth]{images/out/platform_architecture}
    \caption{
        An overview of the stages that flow through different components working together to make up a distributed
        conflict-free replicated abstract syntax tree. Changes detected by the \mintinline{javascript}{Monaco Editor}
        trigger the construction of a new \mintinline{scala}{HeadedAST}, which is then mapped onto the previous
        \mintinline{scala}{HeadedAST} through the \mintinline{scala}{GumTreeAlgorithm} allowing to compute the changes
        that took place as a \mintinline{scala}{MinimumEditScript}.
        These AST changes are serialized and propagated over the wire as \mintinline{scala}{ReplicatedOperation}s, as
        the CRDT is an operation-based CRDT.
        The incoming operations allow each replica to roll back the local AST to then reapply all known changes in order.
        The final computed AST is displayed by the \mintinline{javascript}{Monaco Editor} which then reflects the
        converging AST to the developer.
    }
    \label{fig:platform-architecture}
\end{figure*}

The design of the AST CRDT is highly inspired by the design of the paper "A highly-available move operation for
replicated trees" by~\citet{9563274}.
Their work discusses the design of a tree CRDT, motivating why move operations become complex for replicated trees.
We summarize the discussion on the difficulties:

\begin{itemize}
    \item \textbf{Concurrently moving the same node} \\
    Say concurrently a move operation takes place for the same node, resulting in the node \emph{green} becoming the
    child of its sibling \emph{blue} on one replica and becoming the child of its sibling \emph{orange} on another
    replica as Figure~\ref{fig:tree-move-same} presents.
    Once these operations merge, different operations are possible, prioritize a single outcome
    (Figure~\ref{fig:tree-move-same} \emph{(a)} and \emph{(b)}), or duplicate the \emph{green} node so that both
    \emph{blue} and \emph{orange} can have an instance of \emph{green} as their child
    (Figure~\ref{fig:tree-move-same} \emph{(c)}).
    Another option could be to have both nodes \emph{blue} and \emph{orange} have as their child node \emph{green},
    however this would break the tree-like structure (Figure~\ref{fig:tree-move-same} \emph{(d)}).

    \item \textbf{Introducing cycles} \\
    While move operations may uphold the guarantees of the tree remaining as an acyclic structure on individual
    replica's, their combination may introduce cycles.
    Figure~\ref{fig:tree-move-parent} shows the distinct outcomes possible.
    The figure illustrates the possible choices when two sibling nodes are concurrently made a parent of one another.
    It allows for either move operation to remain (Figure~\ref{fig:tree-move-parent} \emph{(a)} and \emph{(b)}), a
    duplicate instance of both nodes to enable both move operations (Figure~\ref{fig:tree-move-parent} \emph{(c)}) or
    to introduce a cycle to enable both operations without duplication, but this would break the tree requirement
    (Figure~\ref{fig:tree-move-parent} \emph{(d)}).
\end{itemize}

\begin{figure}
    \centering
    \includegraphics[width=0.45\textwidth]{images/out/tree_move_same}
    \caption{
        A case in which concurrent move operations take place for a node to become the child of either one of its
        siblings, including the possible outcomes.
        The shown outcomes illustrate there is either an operation-discarding choice, a duplicating choice or a
        tree-structure breaking choice that must be made.
        Inspired by the illustrations designed by~\citet{9563274}.
    }
    \label{fig:tree-move-same}
\end{figure}

\begin{figure}
    \centering
    \includegraphics[width=0.45\textwidth]{images/out/tree_move_parent}
    \caption{
        A case in which concurrent move operations take place for two nodes to become each others descendant, including
        the possible outcomes.
        The shown outcomes illustrate there is either an operation-discarding choice, a duplicating choice or a
        tree-structure breaking choice that must be made.
        Inspired by the illustrations designed by~\citet{9563274}
    }
    \label{fig:tree-move-parent}
\end{figure}

Their work then proposes to accompany each operation with a timestamp such that incoming operations that are out of
order, allowing the CRDT to determine the point at which these should have been applied.
It then performs a rollback using the \emph{undo\_op} operation up until the point where the incoming operation can be
applied after which all latter operations can be replayed using the \emph{redo\_op} operation.
These are the conditions by which the authors prove convergence in a strong eventual consistent setting.
